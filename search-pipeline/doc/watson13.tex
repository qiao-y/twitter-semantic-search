\documentclass{acm_proc_article-sp}

\begin{document}

\title{Enabling Semantic Searching on Microblogs with Latent Semantic Analysis
\titlenote{This work is done with the help of Prof. Alfio Gliozzo and Dr. Or Biran in partial satisfaction of the requirements of E6998\_9 Spring 2013 course at Columbia University}
}

\numberofauthors{1} 

\author{\alignauthor Wei Wang, Yu Qiao, Qiuzi Shangguan, Ran Yu \\
\affaddr{Department of Computer Science} \\ 
\affaddr{Columbia University}
\affaddr{New York, USA} \\
\email{\{ww2315,yq2145,qs2130,ry2239\}@columbia.edu}
} 
\maketitle

\begin{abstract}
In our project, we implemented a LSA-based semantic search pipeline over tweets. More particularly, we used TREC 2011 microblog track corpus and its evaluation criteria. The results show that our algorithm yield a XXX performance rise compared with baseline using P@30. 

\end{abstract}

\section{Introduction}
Why this project\\
TREC game rules\\

\section{Corpus}
How we get the corpus\\
What it contains\\
The size of the corpus\\

\section{Evaluation Criteria and Baseline}
How TREC evaluates the results\\
How we generated baseline (vs official baseline)\\

\section{Preprocessing}
Remove redundant characters(giiirrl)\\
Remove numbers\\
Remove non-english tweets\\
Dump URLs\\

\section{Rank model}
BM25\\
URL-enhanced ranking formula

\section{LSA}
How distributed LSA works\\

\section{Evaluation Results}
results\\

\section{Conclusion}
What has been done 

\section{Future work}
What needs to be done

\section{Acknowledgments}
Show compliments to Alifo and Or\\  

\bibliographystyle{abbrv}
\bibliography{waton13}
\balancecolumns
\end{document}
